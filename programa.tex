\begin{itemize}
\item \textsc{\textbf{JUEVES 12 DE DICIEMBRE}}
  \begin{itemize}
  \item Sesión de mañana:
    \begin{itemize}
    \item[]9.00-9.30 Recepción y entrega de material
    \item[]9.30-10.00 Inauguración de las jornadas
    \item[]10.00-11.00 Conferencia plenaria:

    ''adabag: An R Package for Classification with Boosting and Bagging'' - 
    \emph{Esteban Alfaro Cortés, Matías Gámez y Noelia García}
    \item[]11.00-11.30 Pausa café
    \item[]11.30-13.30 Sesión de Comunicaciones
    \begin{itemize}
    \item Comunicaciones orales
      \begin{enumerate}
      \item[-] Package xkcd: Plotting ggplot2 graphics in a XKCD style 
       (\emph{Emilio Torres-Manzanera})
      \item[-] Métrica de Wasserstein para la comparación de matrices 
      origen-destino  (\emph{Aleix Ruiz de Villa})
      \item[-]  Categorización automática de contenidos web con R  
      (\emph{Pedro Concejero})
      \item[-] Algunos aspectos prácticos del manejo de datos de encuesta con 
      R  (\emph{Jesús Bouso Freijo})
      \item[-] Sesgo de publicación en ciencias médicas  
      (\emph{Borja Santos Zorrozúa})
      \item[-] Utilidad clínica de modelos predictivos:  análisis mediante 
      funciones de densidad de probabilidad estimadas por métodos tipo 
      kernel   (\emph{Luis Mariano Esteban})
      \item[-] Evaluación del uso de modelos mixtos para estimación de la tasa 
      de paro con poca muestra  (\emph{José Luis Cañadas Reche})
      \end{enumerate}
    \item  Presentaciones breves
      \begin{enumerate}
      \item[-] Mejora de la detección visual de datos atípicos mediante una 
      modificación en las caras de Chernoff  (\emph{Beatriz González Pérez})
      \item[-] Tratamiento de datos con R para control de calidad basado en 
      valoraciones biológicas. Rectas Paralelas  (\emph{Faustino Huertas Muñoz})
      \item[-] Análisis exploratorio de datos del mercado eléctrico español con 
      R  (\emph{J.M.Velasco})
      \end{enumerate}
    \end{itemize}
    \item[]13.30-14.00 Ponencia invitada:

    ''Optimización Entera Mixta No Lineal (MINLP) con R y Pyomo: Un ejemplo 
    práctico'' -- \emph{Jorge Ayuso Rejas}    
  \end{itemize}
  \item Sesión de tarde:
    \begin{itemize}
      \item[]16.00-17.00 Conferencia plenaria:
    
      ''Mejora de la calidad con R: Aplicación de Seis Sigma y otros 
      métodos estadísticos'' -- \emph{Emilio López Cano}
      \item[] 17.00-19.00 Talleres paralelos
      \begin{enumerate}
      \item[-] ''Big data analytics: R + Hadoop'' -- \emph{Carlos Gil Bellosta}
      \item[-] ''Relenium, selenium en R. Un nuevo paquete para webscraping''
       -- \emph{Aleix Ruiz de Villa}
      \end{enumerate}
      \item[]19.00-20.00 Asamblea ''Comunidad R-Hispano''
    \end{itemize}
  
\end{itemize}
 

 

\item \textsc{\textbf{VIERNES 13 DE DICIEMBRE}}
  \begin{itemize}
  \item Sesión de mañana:
    \begin{itemize}
    \item[] 9.00-10.00  Conferencia plenaria: 

    ''Análisis de datos reproducible con R: métodos, herramientas y tendencias''     -- \emph{Felipe Ortega}

    \item[] 10.00-11.00 Mesa redonda: \textsl{Retos ''para y desde'' \texttt{R}}
      \begin{itemize}
      \item Moderador: \emph{Emilio Torres Manzanera}
      \item \emph{Jesús Bouso} (Centro de Investigaciones Sociológicas)
      \item \emph{Santiago Basaldúa} (Synergic Partners)
      \item \emph{Xavier de Blas} (U. Ramon Llul)
      \item \emph{Carlos Gil Bellosta} (R-Hispano)
      \item Representante Open Data Ayuntamiento de Zaragoza
      \end{itemize}
    \item[]11.00-11.30 Pausa café
 
        Punto de encuentro profesional. 
        Cristina Guirado, responsable de recursos humanos de Synergic 
        Partners recogerá curriculums vitae a las personas interesadas.
    \item[]11.30-13.30 Sesión de Comunicaciones
    \begin{itemize}
    \item Comunicaciones orales
      \begin{enumerate}
      \item[-] Desarrollo de Interfaces Web utilizando programación 
      funcional en R (\emph{Jorge Luis Ojeda Cabrera})
      \item[-] Previsión de equipamientos educativos, culturales y sanitarios 
      en los barrios de nueva creación de la ciudad de Zaragoza 
      (\emph{Sergio Jiménez Sanjuán})
      \item[-] El paquete W2CWM2C: análisis de correlación de wavelet. 
      Casos bivariado y multivariado (\emph{Josué M. Polanco Martínez})
      \item[-] Análisis automatizado de cuasi-implicaciones. El Proyecto 
      RCHIC: primeros pasos (\emph{Rubén Pazmiño})
      \item[-] Postprocesado de resultados de analysis de elementos 
      finitos con R (\emph{Andres Sanz-Garcia})
      \item[-] Medición de la potencia en deportistas usando R y 
      encoders (\emph{Xavier de Blas Foix})
      \item[-] Estrategias de captación de clientes en mercados con 
      competencia (\emph{Francisco Jesús Rodríguez Aragón})
      \end{enumerate}
    \item  Presentaciones breves
      \begin{enumerate}
      \item[-] Preprocesado de imágenes hiperespectrales en R 
      (\emph{Rubén Urraca Valle})
      \item[-] Análisis clasificatorio de la actividad electroencefalográfica 
      a través del paso de señales temporales al dominio de la frecuencia 
      (\emph{Roberto Fernandez Martinez})
      \item[-] Simulación en R de modelos definidos en hoja de cálculo 
      (\emph{Ramiro Serrano-García})
      \end{enumerate}
    \end{itemize}
    \item[]13.30-14.00 Ponencia invitada:

    ''Aplicaciones de Big Data en R'' --- \emph{Synergic Partners} 
    \end{itemize}

  \item Sesión de tarde:
    \begin{itemize}
    \item[]16.00-17.00 Conferencia plenaria:
    
    '' Inferencia Bayesiana para modelos espaciales y espacio-temporales 
    con R'' -- \emph{Virgilio Gómez Rubio}
    \item[] 17.00-19.00 Talleres paralelos
    \begin{enumerate}
    \item[-] ''Cazando información espectro-temporal en datos 
             ambientales con R'' -- \emph{Josué M. Polanco Martínez}
    \item[-] ''Docencia de R mediante investigación reproducible. 
              RStudio, knitr, markdown'' -- \emph{José Antonio Palazón Ferrando}
    \end{enumerate}
    \item[]19.00-19:30 Clausura Jornadas
    \end{itemize}
  \end{itemize}
\end{itemize}


