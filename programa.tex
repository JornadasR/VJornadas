\begin{itemize}
\item \textsc{\textbf{JUEVES 15 DE NOVIEMBRE}}
  \begin{itemize}

  \item 09:00-09:30 Acreditación y recogida de información

  \item 09:30-09:45 Inauguración oficial de las Jornadas. J.R. González.

  \item 09:45-10:30 Conferencia Inaugural.  J. Vila:
    \href{http://r-es.org/tiki-download_file.php?fileId=484}{Enseñando
      estadística: como mejorar los conocimientos utilizando R
      para la creación de prácticas individualizadas.}
  \item 10:30-12:00 Sesión de Comunicaciones (I) Moderador: G.R
    Serrano
    \begin{itemize}
    \item 10:30-10:45 C. E. Melo Funciones geoestadísticas y
      funciones de base radial en el programa R: Paquete geospt
    \item 10:45-11:00 E. L. Cano Investigación operativa
      reproducible. Aplicación a la optimización de sistemas
      energéticos
    \item 11:00-11:15 C. J. Gil MicroDatosEs: un paquete para leer
      ficheros de microdatos públicos
    \item 11:15-11:30 A. Alabert Flujo de trabajo reproducible con
      R
    \item 11:30-11:45 N. Longford A study of poverty and income
      inequality in the EU countries
    \end{itemize}
  \item 12:00-12:30 Café
 
  \item 12:30-14:00 Sesión de Comunicaciones (II) Moderador:
    A. Sánchez
    \begin{itemize}
    \item 12:30-12:45 R. Pazmiño Caracterizacion del software
      estadistico en las escuelas de estadistica del
      Ecuador. Enfoque en el software R
    \item 12:45-13:00 O. Ivina A cross-country air quality
      analysis using R

    \item Comunicaciones Breves
      \begin{itemize}
      \item 13:00-13:07 M. Sánchez Inferencia estadística para el
        equilibrio de Hardy-Weinberg en estudios de genotipado con
        Missing Data
      \item 13:07-13:15 I. Roman Representación de las Dinámicas
        de Precios Hoteleros mediante R
      \item 13:15-13:22 D. Moriña El paquete complex.surv.dat.sim
        de R: Simulación de datos de supervivencia complejos
      \item 13:22-13:30 J-L. Cañadas De Excel a html utilizando
        knitr + markdown + googleVis . Un ejemplo
      \item 13:30-13:37 B. González Programación Lineal y
        Programación Dinámica con R
      \item 13:37-13:45 A. Sanz-García Selección de variables y
        modelizado predictivo en R
      \item 13:45-13:52 F. Antoñanzas-Torres Evaluación de modelos
        paramétricos de predicción de irradiación global solar
        mediante variables meteorológicas típicas
      \item 13:52-14:00 R. Fernández Uso de métodos de
        interpolación espacial para la predicción de variables en
        entornos vitivinícolas
      \end{itemize}
    \end{itemize}

  \item 14:00-16:00 Comida
  \item 16:00-17:45 Talleres (I)
    \begin{itemize}
    \item G. R. Serrano Web scraping con R
    \item F. Carmona Informes dinámicos con LaTeX y R: utilización
      de Sweave y knitr.
    \end{itemize}

  \item 17:45-18:15 Café
  \item 18:15-20:00 Talleres (II)
    \begin{itemize}
    \item X. de Pedro Interfaces Web 2.0 para R con Tiki
    \item A. Sánchez Edición (y mucho más) potente en R con ESS
      ("Emacs Speaks Statistics")
    \end{itemize}
  \item 20:00-21:00 Asamblea Asociación “Comunidad R-Hispano”

  \item 21:30 Cena

  \end{itemize}
 

 

\item \textsc{\textbf{VIERNES 16 DE NOVIEMBRE}}

  \begin{itemize}
  \item 10:00-11:00 Sesión de Comunicaciones (III) Moderador:
    F. Carmona
    \begin{itemize}
    \item 10:00-10:15 A. Lobo R como caja de herramientas para SIG
      y Teledetección: reflexiones a partir de experiencias
    \item 10:15-10:30 V. Urrea Gales Simulación de perfiles
      genéticos de riesgo
    \item 10:30-10:45 A. Urkaregi Construcción de un Índice Global
      de Valoración
    \item 10:45-11:00 G. Estévez-Pérez kerdiest: An R Package for
      Distribution Function Estimation and Applications
    \end{itemize}

  \item 11:00-12:00 Sesión de Comunicaciones (IV) Moderador:
    Ll. Ramon
    \begin{itemize}
    \item 11:00-11:15 N. M. Villanueva seq2R: Detección de puntos
      de cambio en secuencias genómicas
    \item 11:15-11:30 J Graffelman Exploring bi-allelic genetic
      markers with the HardyWeinberg package
    \item 11:30-11:45 M. Sestelo FWDselect: Selección de variables
      en modelos de regresión
    \item 11:45-12:00 B. Santos Reducción unidimensional de 12
      items de la Escala de sobrecarga de Zarit en cuidadores de
      pacientes con demencia mediante teoría de respuesta a los
      ítems.
    \item 12:00-12:15 J. Barrera The optimal Allocation package
      for longitudinal studies design with time-varying esposure
    \end{itemize}
  	 
  \item 12:15-12:45 Café

  \item 12:45-14:30 Talleres (III)
    \begin{itemize}
    \item A. Karatzoglou Machine learning in R
    \end{itemize}


  \item 14:30-16:15 Comida
  \item 16:15-18:00 Talleres (IV)
    \begin{itemize}
    \item A. Ruiz Introducción a las Reference Classes
      (programación orientada a objetos en R)
    \item Ll. Ramon, R. Borras y A. Vall Introducción práctica a
      la librería ggplot2 y su integración con ggmap
    \end{itemize}
  \item 18:00-18:30 Café

  \item 18:30-19:00 Clausura Oficial de las IV Jornadas

  \end{itemize}
\end{itemize}


