\chapter{Relenium, selenium en R. Un nuevo paquete para webscraping.}

\chapterprecis{Aleix Ruiz de Villa, Lluis Ramon, Andreu Vall\\TSS - Transport Simulation Systems\\RugBcn, Grupo de usuarios de Barcelona}

\index{Ruiz de Villa, Aleix}
\index{Ramon, Lluis}
\index{Vall, Andreu}

\index[inst]{TSS - Transport Simulation Systems}
\index[inst]{RugBcn, Grupo de usuarios de Barcelona}

Actualmente, los paquetes más utilizados para hacer web scraping con R són XML y RCurl. Ambos permiten 'parsear' el código html de la página web y extraer la información que nos interese. Sin embargo, ninguno de ellos permite interactuar con los elementos javascript de la página. Por tanto aquella información que dependa de la ejecución de comandos javascript (por ejemplo, abrir una ventana con una dirección url desconocida, o seleccionar elementos en un menú desplegable) queda inaccesible.

Relenium es un importador del módulo Selenium de java, via rJava. Selenium nació para el testeo automático de páginas web. La diferencia principal con los paquetes descritos anteriormente es que Relenium puede emular la navegación de un usuario humano, es decir, apretar botones, seleccionar menús, etc. El resultado es una navegación por la web intuitiva y sencilla.

En este taller, introduciremos los elementos básicos del lenguage html y los xpaths, y mostraremos las funcionalidades básicas del paquete relenium. Lo complementaremos con las funcionalidades básicas de XML. No es necesario ningún conocimento previo. 



%\bibliographystyle{plain}

%\bibliography{talleres/relenium__selenium_en_r__un_nuevo_paquete_para_webscraping_}
