\chapter{Big data analytics: R + Hadoop}

\chapterprecis{Carlos J. Gil Bellosta\\Datanalytics (http://www.datanalytics.com/)}

\index{J. Gil Bellosta, Carlos}

\index[inst]{Datanalytics (http://www.datanalytics.com/)}

El taller es una introducción al análisis de datos masivos almacenados
en Hadoop con R utilizando, principalmente, el paquete rmr2. Este
paquete permite distribuir tareas paralelizables en distintos nodos
para procesar conjuntos de datos que no pueden analizarse en memoria.

Una de las operaciones más básicas que cubrirá el taller es la de
contar ocurrencias. Pero también se prestará atención a operaciones
más propias de R, tales como construir modelos y realizar
predicciones.

Finalmente, se utilizará \textit{hadoop streaming} para realizar
simulaciones masivas en paralelo. Este ejemplo servirá, además, para
ilustrar los mecanismos internos del paquete rmr2 y del funcionamiento
de Hadoop.

Los asistentes al taller aprenderán qué es Hadoop, las operaciones
básicas del sistema de ficheros, a crear sus propios procesos
\textit{mapreduce} y, particularmente, a comprender el funcionamiento
del sistema de paralelización de tareas.


%\bibliographystyle{plain}

%\bibliography{talleres/big_data_analytics__r___hadoop}
