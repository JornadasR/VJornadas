\part{Información General}

\textbf{TODO: Vas por aquí}
\chapter{Presentación}

Las V Jornadas de Usuarios de R tendrán lugar en el
\href{http://www.zaragoza.es/ciudad/idezar/detalle_Centro?id=5105}{Etopia-Centro
  de Arte y Tecnología de Zaragoza}, los días 12 y 13 de diciembre de
2013. Etopia es un centro de creatividad, innovación y emprendimiento,
16.000 m2 para el trabajo colaborativo, la búsqueda de nuevos caminos
y para aprender haciendo y compartiendo. Forma parte de la
\href{http://www.zaragoza.es/ciudad/sectores/tecnologia/milladigital.htm}{Milla
  Digital de Zaragoza}, promovida por el
\href{http://www.zaragoza.es/ciudad/sectores/tecnologia/}{Área de
  Tecnología del ayuntamiento de Zaragoza}.

Las jornadas, como no podría ser de otra forma, van a incluir trabajos
de todos los ámbitos y están abiertas tanto a usuarios como a
entusiastas de R independientemente de su área de interés. Los
objetivos para estas jornadas serán los mismos que para las anteriores
que tan buenos resultados obtuvieron. Estos objetivos incluyen:

\begin{itemize}
\item Proporcionar un punto de encuentro a los usuarios de R 
\item Fomentar la colaboración entre ellos en un ambiente multidisciplinar 
\item Divulgar el conocimiento del lenguaje y sus posibilidades 
\item Promover el uso de R 
\end{itemize}

Usuarios y entusiastas de R de todos los ámbitos —universidad,
institutos de investigación, administraciones públicas, empresa
privada— están invitados a participar en las V Jornadas y compartir
con la comunidad aplicaciones y ejemplos interesantes que reflejen la
madurez de R y la diversidad de los problemas y campos en los que
viene utilizándose con éxito. Existen las siguientes modalidades de
participación:
\begin{itemize}
\item Comunicaciones orales de 15 minutos seguidas de una discusión de
  5 minutos (la decisión sobre la duración podría sufrir
  modificaciones en función del número final de ellas).

\item Presentaciones breves de 5 minutos, donde el ponente expone en
  tres diapositivas (número orientativo) de forma breve y concisa,
  quién es/son, qué ha/n hecho, y qué resultados y conclusiones se
  extraen de ello que puedan ser de interés para otras personas.

\item Talleres de 2 horas aproximadamente, donde se explican paquetes,
  procedimientos, y programas de R.  En esta edición, además de las
  ponencias invitadas, las presentaciones orales y los talleres, se
  llevarán a cabo presentaciones breves donde el ponente expondrá de
  forma concisa los resultados y conclusiones de alguna investigación
  llevada a cabo con R que puedan ser de interés para otros colegas.
\end{itemize}

Desde el comité organizador nos gustaría destacar la excelente labor
llevada a cabo por el comité científico, a los ponentes de los
talleres y a todos los asistentes que han permitido confeccionar el
programa que a continuación detallamos y esperamos que sea de vuestro
interés.

Esperamos que las jornadas resulten lo más provechosas posibles y que
disfrutéis de una confortable estancia en Zaragoza.


\chapter{Información útil}

\section{Ubicación de las jornadas}

Las jornadas se celebrarań en el Centro de Arte y Tecnología 
\href{http://www.zaragoza.es/ciudad/idezar/detalle_Centro?id=5105}{ETOPIA}
que  el Ayuntamiento de Zaragoza ha desarrollado en la llamada 
\href{http://www.milladigital.es/espanol/home.php}{Milla Digital}. 

\begin{center}
\includegraphics[width=0.6\textwidth]{Logos/logoMillaAyZgz.png}
\end{center}

El C.A.T. está situado prácticamente en el centro de la Milla Digital, 
justo enfrente de la Estación de Delcias.  se encuentra comunicado con centro de la ciduad mediante las lineas de autobuses
\textbf{TODO: Poner lineas de autobuses y mapa de paradas señalando
la estación rel tryp, etc...}

Las comunicaciones orales y breves se llevarán a cabo en el 
\textbf{Auditorio del CAT ** TODO: nombre y situación de la sala}. 

Para acceder al edificio cada participante se deberá identificar en
recepción donde disponen de una lista con todos los asistentes.


\section{Talleres}

Los participantes a los talleres deben traer su propio ordenador
portátil con las herramientas que indiquen los responsables de
talleres. \textbf{TODO: cómo se desarrollará la inscripció a los
  talleres} La inscripción de los talleres se realizará tal y como
indica la web de las jornadas
\href{http://r-es.org/V+Jornadas#Talleres}.  Dado el limitado número
de plazas, se reservará plaza por orden de inscripción. Los talleres
se desarrollarán en \textbf{Aulas de los laboratorios del CAT ** TODO:
  nombre y situación de las salas}, ambas situadas en el espacio que
reserva ETOPIA para laboratorios audiovisuales y meetings y que serán
convenientemente señal.


\section{Certificados}
Los certificados se enviarán por correo electrónico una vez pasadas
las Jornadas.  

\section{Material}

Todo el material, está disponible a través de la página web de las Jornadas 
\href{http://r-es.org/V+Jornadas}. 



\chapter{Comité organizador}

\begin{itemize}

\item \href{http://www.datanalytics.com}{J. Gil Bellosta} (coordinador)
\item \href{http://www.scien-analytics.com}{Sergio Jiménez} (Scien Analytics)
\item Luis Mariano Esteban (U. de Zaragoza)
\item \href{http://www.scien-analytics.com}{Rubén Moreno Ruíz} (Scien Analytics)
\item \href{[http://www.scien-analytics.com}{Miguel Ángel Luzón} (Scien Analytics)
\item Jorge Ojeda (U. de Zaragoza)
\item \href{http://ueb.vhir.org|Vall d'Hebron Research Institute}{Xavier de Pedro Puente}
\item  Emilio Torres Manzanera (U. de Oviedo)
\end{itemize}

\chapter{Comité científico}


\begin{itemize}

\item Sandra Barragán
\item Ramón Díaz-Uriarte
\item Juan Ramon González
\item \href{http://oscarperpinan.github.io}{Oscar Perpiñán}
\item Miguel Angel Rodríquez (coordinador)
\item Isaac Subirana
\item Joan Vila
\item \href{http://www.ottofwagner.es}{Otto F. Wagner}

\end{itemize}


\chapter{Patrocinadores}



\begin{center}

\includegraphics[width=0.33\textwidth]{Logos/logoMillaAyZgz.png}
\hspace{1cm}

\includegraphics[width=0.33\textwidth]{Logos/logoEtopia.jpg}
\hspace{1cm}

\includegraphics[width=0.33\textwidth]{Logos/logoScien.png}
\hspace{1cm}

\includegraphics[width=0.33\textwidth]{Logos/logoComRHisp.png}
\vspace{1cm}

\includegraphics[width=0.33\textwidth]{Logos/logoRevolAnal}
\hspace{1cm}

\includegraphics[width=0.33\textwidth]{Logos/logoUZ.png}
\hspace{1cm}

\includegraphics[width=0.33\textwidth]{Logos/logoSynergic.png}
\vspace{1cm}

\includegraphics[width=0.33\textwidth]{Logos/logoTelefID.png}

\end{center}


\chapter{Programa}

% \begin{itemize}
\item \textsc{\textbf{JUEVES 12 DE DICIEMBRE}}
  \begin{itemize}
  \item Sesión de mañana:
    \begin{itemize}
    \item[]9.00-9.30 Recepción y entrega de material
    \item[]9.30-10.00 Inauguración de las jornadas
    \item[]10.00-11.00 Conferencia plenaria:

    ''adabag: An R Package for Classification with Boosting and Bagging'' - 
    \emph{Esteban Alfaro Cortés, Matías Gámez y Noelia García}
    \item[]11.00-11.30 Pausa café
    \item[]11.30-13.30 Sesión de Comunicaciones
    \begin{itemize}
    \item Comunicaciones orales
      \begin{enumerate}
      \item[-] Package xkcd: Plotting ggplot2 graphics in a XKCD style 
       (\emph{Emilio Torres-Manzanera})
      \item[-] Métrica de Wasserstein para la comparación de matrices 
      origen-destino  (\emph{Aleix Ruiz de Villa})
      \item[-]  Categorización automática de contenidos web con R  
      (\emph{Pedro Concejero})
      \item[-] Algunos aspectos prácticos del manejo de datos de encuesta con 
      R  (\emph{Jesús Bouso Freijo})
      \item[-] Sesgo de publicación en ciencias médicas  
      (\emph{Borja Santos Zorrozúa})
      \item[-] Utilidad clínica de modelos predictivos:  análisis mediante 
      funciones de densidad de probabilidad estimadas por métodos tipo 
      kernel   (\emph{Luis Mariano Esteban})
      \item[-] Evaluación del uso de modelos mixtos para estimación de la tasa 
      de paro con poca muestra  (\emph{José Luis Cañadas Reche})
      \end{enumerate}
    \item  Presentaciones breves
      \begin{enumerate}
      \item[-] Mejora de la detección visual de datos atípicos mediante una 
      modificación en las caras de Chernoff  (\emph{Beatriz González Pérez})
      \item[-] Tratamiento de datos con R para control de calidad basado en 
      valoraciones biológicas. Rectas Paralelas  (\emph{Faustino Huertas Muñoz})
      \item[-] Análisis exploratorio de datos del mercado eléctrico español con 
      R  (\emph{J.M.Velasco})
      \end{enumerate}
    \end{itemize}
    \item[]13.30-14.00 Ponencia invitada:

    ''Optimización Entera Mixta No Lineal (MINLP) con R y Pyomo: Un ejemplo 
    práctico'' -- \emph{Jorge Ayuso Rejas}    
  \end{itemize}
  \item Sesión de tarde:
    \begin{itemize}
      \item[]16.00-17.00 Conferencia plenaria:
    
      ''Mejora de la calidad con R: Aplicación de Seis Sigma y otros 
      métodos estadísticos'' -- \emph{Emilio López Cano}
      \item[] 17.00-19.00 Talleres paralelos
      \begin{enumerate}
      \item[-] ''Big data analytics: R + Hadoop'' -- \emph{Carlos Gil Bellosta}
      \item[-] ''Relenium, selenium en R. Un nuevo paquete para webscraping''
       -- \emph{Aleix Ruiz de Villa}
      \end{enumerate}
      \item[]19.00-20.00 Asamblea ''Comunidad R-Hispano''
    \end{itemize}
  
\end{itemize}
 

 

\item \textsc{\textbf{VIERNES 13 DE DICIEMBRE}}
  \begin{itemize}
  \item Sesión de mañana:
    \begin{itemize}
    \item[] 9.00-10.00  Conferencia plenaria: 

    ''Análisis de datos reproducible con R: métodos, herramientas y tendencias''     -- \emph{Felipe Ortega}

    \item[] 10.00-11.00 Mesa redonda: \textsl{Retos ''para y desde'' \texttt{R}}
      \begin{itemize}
      \item Moderador: \emph{Emilio Torres Manzanera}
      \item \emph{Jesús Bouso} (Centro de Investigaciones Sociológicas)
      \item \emph{Santiago Basaldúa} (Synergic Partners)
      \item \emph{Xavier de Blas} (U. Ramon Llul)
      \item \emph{Carlos Gil Bellosta} (R-Hispano)
      \item Representante Open Data Ayuntamiento de Zaragoza
      \end{itemize}
    \item[]11.00-11.30 Pausa café
 
        Punto de encuentro profesional. 
        Cristina Guirado, responsable de recursos humanos de Synergic 
        Partners recogerá curriculums vitae a las personas interesadas.
    \item[]11.30-13.30 Sesión de Comunicaciones
    \begin{itemize}
    \item Comunicaciones orales
      \begin{enumerate}
      \item[-] Desarrollo de Interfaces Web utilizando programación 
      funcional en R (\emph{Jorge Luis Ojeda Cabrera})
      \item[-] Previsión de equipamientos educativos, culturales y sanitarios 
      en los barrios de nueva creación de la ciudad de Zaragoza 
      (\emph{Sergio Jiménez Sanjuán})
      \item[-] El paquete W2CWM2C: análisis de correlación de wavelet. 
      Casos bivariado y multivariado (\emph{Josué M. Polanco Martínez})
      \item[-] Análisis automatizado de cuasi-implicaciones. El Proyecto 
      RCHIC: primeros pasos (\emph{Rubén Pazmiño})
      \item[-] Postprocesado de resultados de analysis de elementos 
      finitos con R (\emph{Andres Sanz-Garcia})
      \item[-] Medición de la potencia en deportistas usando R y 
      encoders (\emph{Xavier de Blas Foix})
      \item[-] Estrategias de captación de clientes en mercados con 
      competencia (\emph{Francisco Jesús Rodríguez Aragón})
      \end{enumerate}
    \item  Presentaciones breves
      \begin{enumerate}
      \item[-] Preprocesado de imágenes hiperespectrales en R 
      (\emph{Rubén Urraca Valle})
      \item[-] Análisis clasificatorio de la actividad electroencefalográfica 
      a través del paso de señales temporales al dominio de la frecuencia 
      (\emph{Roberto Fernandez Martinez})
      \item[-] Simulación en R de modelos definidos en hoja de cálculo 
      (\emph{Ramiro Serrano-García})
      \end{enumerate}
    \end{itemize}
    \item[]13.30-14.00 Ponencia invitada:

    ''Aplicaciones de Big Data en R'' --- \emph{Synergic Partners} 
    \end{itemize}

  \item Sesión de tarde:
    \begin{itemize}
    \item[]16.00-17.00 Conferencia plenaria:
    
    '' Inferencia Bayesiana para modelos espaciales y espacio-temporales 
    con R'' -- \emph{Virgilio Gómez Rubio}
    \item[] 17.00-19.00 Talleres paralelos
    \begin{enumerate}
    \item[-] ''Cazando información espectro-temporal en datos 
             ambientales con R'' -- \emph{Josué M. Polanco Martínez}
    \item[-] ''Docencia de R mediante investigación reproducible. 
              RStudio, knitr, markdown'' -- \emph{José Antonio Palazón Ferrando}
    \end{enumerate}
    \item[]19.00-19:30 Clausura Jornadas
    \end{itemize}
  \end{itemize}
\end{itemize}




