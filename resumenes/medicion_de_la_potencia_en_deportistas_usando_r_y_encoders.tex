\chapter{Medición de la potencia en deportistas usando R y encoders}

\chapterprecis{Xavier de Blas Foix\\Universitat Ramon Llull\\FPCEE Blanquerna\\Grupo SAFE\\Chronojump-Boscosystem}

\index{de Blas Foix, Xavier}

\index[inst]{Universitat Ramon Llull}
\index[inst]{FPCEE Blanquerna}
\index[inst]{Grupo SAFE}
\index[inst]{Chronojump-Boscosystem}

La medición de la fuerza en los deportistas se ha realizado tradicionalmente a partir de observar la máxima carga que éstos pueden levantar, sin ir ligado ello a velocidad, aceleración o potencia. En los últimos años han aparecido en el mercado algunos codificadores (encoders) que calculan la potencia para cada carga levantada, siendo un parámetro mucho más relevante en la mayoría de los deportes, y permitiendo conocer si se está entrenando correctamente. Estos encoders tienen un coste económico alto y no son software libre.

En la comunicación se presentan tres modelos de encoder que pueden conectarse a una placa de hardware libre: Chronopic y un firmware y software de captura y gestión libres. Las piezas de software analizan los datos que proceden del encoder usando scripts de R. El conjunto se conecta al software Chronojump, un software libre que desde hace varios años se comunica con R para sus cálculos.
 \bigskip\subsection*{Bibliografía}

 1. De Blas Foix, F. X. (2012). Proyecto Chronojump-Boscosystem. Herramienta informatica libre para el estudio cinematico del salto vertical: medicion del tiempo, deteccion del angulo de flexion sin marcadores y elaboracion de tablas de percentiles.



2. Gonzalez-Badillo, J.G., y Sanchez-Medina, L.S. (2010). Movement velocity as a measure of loading intensity in resistance training. Int J Sports Med, 31, 347-352.



3. Padulles, J.M. (2011). Valoracion de los parametros mecanicos de la carrera. Desarrollo de un nuevo instrumento de medicion. Tesis doct. Barcelona: INEFC, Universitat de Barcelona.



4. Pena, J. (2013). El entrenamiento de la condicion fisica en el voleibol. Fundacion CIDIDA.



5. Tous Fajardo, J. (1999). Nuevas tendencias en fuerza y musculacion.

%\bibliographystyle{plain}

%\bibliography{resumenes/medicion_de_la_potencia_en_deportistas_usando_r_y_encoders}
