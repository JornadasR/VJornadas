\chapter{Análisis exploratorio de datos del mercado eléctrico español con R}

\chapterprecis{J.M. Velasco, B. González, G. Miñana, R. Caro, H. Marrao, J. Gil, V. López\\Departamento de Arquitectura de computadores y automática. Universidad Complutense de Madrid.\\Indizen Technologies, S.L.}

\index{Velasco, J.M.}
\index{González, B.}
\index{Miñana, G.}
\index{Caro, R.}
\index{Marrao, H.}
\index{Gil, J.}
\index{López, V.}

\index[inst]{Departamento de Arquitectura de computadores y automática. Universidad Complutense de Madrid.}
\index[inst]{Indizen Technologies, S.L.}

En este trabajo se presenta un análisis exploratorio de datos desarrollado con R,  aplicado al mercado eléctrico español. Se han utilizado los datos públicos de los años 2011 y 2012 disponibles en www.omelholding.es. En primer lugar se introducen los conceptos necesarios para comprender el mercado de la energía en España, así como las características esenciales sobre este recurso no almacenable. Una vez definidas las variables de interés, se analizan formas para medir tanto la oferta como la demanda y de todo ello se infiere el precio en el mercado. Para poder realizar un modelo matemático correcto, se requiere de un análisis de los datos previo donde se determinen las dependencias entre las variables, las correlaciones, los valores atípicos y la normalidad de las variables. Este estudio se ha realizado con R mediante programas fuentes incluidos en el anexo y funciones específicas de librerías de R que también se enumeran y se comentan en el trabajo. Los resultados son de dos tipos: numéricos y gráficos. La gran cantidad de gráficos ofrecen al lector una mejor visualización de los datos y por tanto una mejor interpretación de los resultados.

%\bibliographystyle{plain}

%\bibliography{resumenes/analisis_exploratorio_de_datos_del_mercado_electrico_espanol_con_r}
