\chapter{Docencia de R mediante investigación reproducible. `RStudio`, `knitr`, `markdown` }

\chapterprecis{Jose Antonio Palazon Ferrando y Antonio  Maurandi López\\Universidad de Murcia. Comunidad R-Hispano.\\Departamento de Ecología e Hidrología. Facultad de Biología.\\Sec. Apoyo Estadístico. Servicio de Apoyo a la Investigación (SAI)}

\index{Antonio Palazon Ferrando y Antonio  Maurandi López, Jose}

\index[inst]{Universidad de Murcia. Comunidad R-Hispano.}
\index[inst]{Departamento de Ecología e Hidrología. Facultad de Biología.}
\index[inst]{Sec. Apoyo Estadístico. Servicio de Apoyo a la Investigación (SAI)}

La utilización de la metodología de enseñanza basada en problemas puede reforzarse, en el caso del uso de R, con la disponibilidad de herramientas para elaborar documentos de calidad y con vocación reutilizable.

La combinación  `RStudio`

%\bibliographystyle{plain}

%\bibliography{resumenes/docencia_de_r_mediante_investigacion_reproducible___rstudio____knitr____markdown__}
