\chapter{Desarrollo de Interfaces Web utilizando programación funcional en R}

\chapterprecis{Jorge~Luis Ojeda Cabrera\\Dept- Métodos Estadísticos, Univ. de Zaragoza}

\index{Ojeda Cabrera, Jorge~Luis}

\index[inst]{Dept- Métodos Estadísticos, Univ. de Zaragoza}

Este trabajo muestra el desarrollo de interfaces web para funciones en R mediante las ideas utilizadas en el paquete 'miniGUI'. Tanto en dicho paquete como en este trabajo se propugna el uso de las capacidades de R para desarrollar programación funcional y 'calcular sobre el lenguaje' a fin de disociar el código necesario para desarrollar los  cálculos puramente estadísticos del código utilizado en la construcción de la interfaz 
de usuario. Esto no sólo ayuda al desarrollo rápido de aplicaciones web, sino que permite separar convenientemente y de una forma sencilla la construcción del Interfaz de la funcionalidad estadística, proporcionando además completa flexibilidad a la hora de desarrollar los interfaces.

  En este caso se desarrollan Interfaces Web para el usuario (WUI)en HTML para 
funciones R que permiten la introducción de los datos mediante formularios HTML. 
El paquete ha sido probado con la utilidad CGI R FastRWeb y con la aplicación 
web sumo con configuración básica.

  El desarrollo de este trabajo se concreta de momento en una versión incompleta
del paquete miniHtmlWUI en la que se implementan todas estas ideas junto 
con algunos ejemplos básicos de la misma.

%\bibliographystyle{plain}

%\bibliography{resumenes/desarrollo_de_interfaces_web_utilizando_programacion_funcional_en_r}
