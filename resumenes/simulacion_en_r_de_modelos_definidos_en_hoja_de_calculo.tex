\chapter{Simulación en R de modelos definidos en hoja de cálculo}

\chapterprecis{Ramiro Serrano García, Gregorio~R. Serrano\\Keller Graduate School of Management\\Universidad Complutense de Madrid}

\index{Serrano García, Ramiro}
\index{Serrano, Gregorio~R.}

\index[inst]{Keller Graduate School of Management}
\index[inst]{Universidad Complutense de Madrid}

Presentamos un complemento de Excel para realizar simulación de Montecarlo en R sobre modelos definidos en hoja de cálculo. Con la aplicación (Stochastic-e) se identifican y gestionan las variables del modelo, se definen los parámetros de la simulación y el conjunto de resultados. En cambio, es en R donde se generan los números aleatorios y se realizan los cálculos y análisis estadísticos definidos por el usuario antes de ser devueltos a la hoja de cálculo. Utilizamos el paquete
XLConnect, lo que permite adecuar Stochastic-e para su uso con otras hojas de cálculo. Con esta estrategia, el coste de aprendizaje se reduce y la herramienta es accesible para estudiantes de distintas disciplinas mientras se mantiene un
elevado nivel de rigor estadístico. \bigskip\subsection*{Bibliografía}

 1. Baier T, Neuwirth E and Meo MD (2011). "Creating and Deploying an Application with (R)Excel and R." The R Journal, *3*(2), pp. 5-11.

2. Davis FD (1989). "Perceived usefulness, perceived ease of use and user acceptance of information technology." MIS Quarterly, *13*(3), pp. 319-340. <URL: http://www.jstor.org/ pss/249008>.

3. Heiberger RM and Neuwirth E (2009). R Through Excel: A Spreadsheet Interface for Statistics, Data Analysis, and Graphics. Springer, New York.

4. McCullough BD and Wilson B (2002). "On the accuracy of statistical procedures in Microsoft Excel 2000 and Excel XP." Computational Statistics and Data Analysis, pp. 713-721.

5. Seila AF (2005). "Simulation Conference, 2005 Proceedings of the Winter." In Georgia Univ, pp. 7803-9519.



%\bibliographystyle{plain}

%\bibliography{resumenes/simulacion_en_r_de_modelos_definidos_en_hoja_de_calculo}
