\chapter{Sesgo de publicación en ciencias médicas}

\chapterprecis{ Borja Santos Zorrozúa1, 2, 3, Eduardo González Fraile4, , Javier Ballesteros Rodríguez2, 4\\1 Universidad del País Vasco (UPV/EHU), 2 Cibersam (G16), 3 Programa PREDOC Gobierno Vasco, 4 Instituto de Investigaciones Psiquiátricas}

\index{Borja Santos Zorrozúa1, }
\index{, 2}
\index{, 3}
\index{González Fraile4, Eduardo}
\index{, NA}
\index{Ballesteros Rodríguez2, Javier}
\index{, 4}

\index[inst]{1 Universidad del País Vasco (UPV/EHU), 2 Cibersam (G16), 3 Programa PREDOC Gobierno Vasco, 4 Instituto de Investigaciones Psiquiátricas}

El metaanálisis es un herramienta muy utilizada en las ciencias médicas para relaizar una síntesis de la evidencia científica publicada relacionada con un mismo tema. A pesar de ser una técnica depurada, cuenta con posibles limitaciónes y errores sitemáticos. 

El sesgo de publicación supone una de sus mayores limitaciones. Se define como la no publicación de manera deliberada de estudios no favorables a las hipótesis establecidas previamente. Los motivos de este fenómeno pueden ser entre otros: intereses comerciales de medicamentos, falta de interés de publicación por parte del investigador independiente, limitaciones idiomáticas o de localización, o limitaciones editoriales.

La existencia de este sesgo se traduce en una estimación errónea del tamaño del efecto combinado de varios estudios (los trazados y publicados). Es por esto que existen diferentes técnicas para ajustar el tamaño del efecto combinado asumiendo la existencia de dicho sesgo. 

El objetivo de esta presentación es probar el funcionamiento de las diferentes librerias existentes en R que permiten ajustar por la existencia de sesgo de publicación: meta, metafor, Copas, SAMURAI, selectMeta. Para ello utilizaremos una serie de estudios que analizan la efectividad de la agomelatina como tratamiento de la depresión. Este conjunto está formado por estudios ya publicados (corroboran la eficacia de este tratamiento) y de otros que no han sido publicados (debido a sus pobres resultados).

De esta manera como hemos tenido la posibilidad de metaanalizar la totalidad de estudios, conocemos el verdadero tamaño del efecto de la agomelatina. Por lo tanto enfrentaremos a este, los estimadores del tamaño del efecto obtenidos al poner en práctica las librerías mencionadas anteriormente y de este modo conocer cual es su precisión a la hora de calcular el tamaño del efecto.




%\bibliographystyle{plain}

%\bibliography{resumenes/sesgo_de_publicacion_en_ciencias_medicas}
