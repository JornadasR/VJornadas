\chapter{Análisis clasificatorio de la actividad electroencefalográfica a través del paso de señales temporales al dominio de la frecuencia}

\chapterprecis{Roberto Fernandez Martinez, Ruben Lostado Lorza, Ruben Urraca Valle, Andres Sanz Garcia\\Department of Mining and Metallurgical Engineering and Materials Science. University of Basque Country. Spain\\Universidad de La Rioja. Spain\\Universidad de La Rioja. Spain\\Division of Bioscience. University of Helsinki. Finland}

\index{Fernandez Martinez, Roberto}
\index{Lostado Lorza, Ruben}
\index{Urraca Valle, Ruben}
\index{Sanz Garcia, Andres}

\index[inst]{Department of Mining and Metallurgical Engineering and Materials Science. University of Basque Country. Spain}
\index[inst]{Universidad de La Rioja. Spain}
\index[inst]{Universidad de La Rioja. Spain}
\index[inst]{Division of Bioscience. University of Helsinki. Finland}

Esta comunicación presenta la primera parte del trabajo realizado para clasificar los diferentes estados o sentimientos que una persona puede tener al realizar ciertas acciones. Se muestra cómo mediante la utilización de un EGG (encefalograma) multicanal se pueden clasificar las emociones que una persona tiene al visionar varios videos. Se analizan diferentes estados como pueden ser emoción y sorpresa, felicidad y placer, logro y compromiso, confusión y desconcierto, y aburrimiento. A través del uso de un EGG se obtienen valores que captan las pequeñas señales eléctricas que las células del cerebro humano producen al comunicarse entre ellas. Posteriormente se convierten las señales recogidas por los 14 canales del EGG al dominio de la frecuencia, utilizando las conocidas técnicas de análisis de Fourier y además diferentes tipos de filtros a la hora de adecuar la señal. Las señales recogidas son filtradas para eliminar ruidos y posteriormente obtener las siguientes variables significativas que según la literatura definen los cambios de energía: banda alfa (8-13 Hz), banda delta (0-4 Hz), banda beta (14-60 Hz) y banda theta (4-7 Hz). Una vez conocidos las bandas en cada situación se realiza un análisis de la varianza para conocer como de precisa puede ser la futura clasificación de los diferentes estados. Para ellos cuatro test de análisis de varianza son utilizados: ANOVA, Bartlett test, Brown-Forsyth test y Fligner-Killeen test. Se analizan los cuatro test para cubrir los casos de variables paramétricas, semi-paramétricas y no paramétricas. Con este análisis se confirma si la hipótesis nula puede ser rechazada y además se conoce cuanto de diferentes pueden ser las clases estudiadas.

%\bibliographystyle{plain}

%\bibliography{resumenes/analisis_clasificatorio_de_la_actividad_electroencefalografica_a_traves_del_paso_de_senales_temporales_al_dominio_de_la_frecuencia}
