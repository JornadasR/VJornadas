\chapter{Previsión de equipamientos educativos, culturales y sanitarios en los barrios de nueva creación de la ciudad de Zaragoza}

\chapterprecis{Sergio Jiménez Sanjuán\\SCIEN Analytics}

\index{Jiménez Sanjuán, Sergio}

\index[inst]{SCIEN Analytics}

El objetivo fundamental del estudio es hacer una previsión de necesidades futuras de equipamientos para el horizonte temporal 2013- 2022 en los barrios de nueva creación  de la ciudad de Zaragoza. 



El primer objetivo es la estimación  de la población futura de los barrios de nueva creación de la ciudad de Zaragoza. Los barrios a estudiar presentan diferentes problemáticas a la hora de analizar su dinámica poblacional por lo que requerirán métodos y técnicas diferenciadas.



El otro pilar del proyecto es determinar la población a la que es capaz de dar servicio un equipamiento. Responderemos a esta cuestión desde un punto de vista práctico. Determinaremos la población típica a la que están dando servicio, en la actualidad, los distintos tipos de equipamientos que abarca el estudio siguiendo estos pasos:



\begin{itemize}
\item Calcular las áreas de influencia de los distintos equipamientos
\item Calcular la población total, y composición, que vive dentro de cada área de influencia
\item Estudiar estadísticamente las distribuciones de  poblacion de todas las áreas de influencia y calcular unos intervalos de población típicos a los que están dando servicio los equipamientos en la actualidad
\end{itemize} 

Finalmente utilizaremos un  criterío  de mínimos respecto a las necesidades futuras. Es decir, supondremos necesarios un número de equipamientos tal que teniendo en cuenta la población prevista a la que daría cobertura cada equipamiento se situara entre el percentil 75 y 90 de los que atienden a mayor número población en la actualidad (2012).


El objetivo de la ponencia, además de  la presentación de los resultados del estudio, es ilustrar el uso de R y de los diferentes paquetes que se ha realizado en su desarrollo:
\begin{itemize}
\item Desarga y análisis de datos INE: paquete pxR
\item Procesado de cartografías manzana a manzana: PBSmapping, maptools
\item Descarga de datos de equipamientos: RJSON, XML
\item Cálculo de areas de influencia de equipamientos: PBSmapping, rgdal
\item Análisis de Datos de población y previsión de población futura 
\item Previsión de población por franjas de edades 
\item Mapas: ggmap
\end{itemize} \bigskip\subsection*{Bibliografía}

 1. Estadistica INd (????). Proyeccion de la Poblacion de Espana a Corto Plazo (2011-2021). Metodologia..

2. Viciana FJ, Gil Bellosta C and Perpinan Lamigueiro O (2011). pxR: PC-Axis with R. R package version 0.24, <URL: http://CRAN.R-project.org/package=pxR>.

3. Schnute JT, Boers N and Haigh. R (2012). PBSmapping: Mapping Fisheries Data and Spatial Analysis Tools. R package version 2.62.34, <URL: http://CRAN.R-project.org/ package=PBSmapping>.

4. Keitt TH, Bivand R, Pebesma E and Rowlingson B (2012). rgdal: Bindings for the Geospatial Data Abstraction Library. R package version 0.7-12, <URL: http://CRAN.R-project.org/package=rgdal>.

5. Kahle D and Wickham H (2012). ggmap: A package for spatial visualization with Google Maps and  OpenStreetMap. R package version 2.1, <URL: http://CRAN.R-project.org/package=ggmap>.



%\bibliographystyle{plain}

%\bibliography{resumenes/prevision_de_equipamientos_educativos__culturales_y_sanitarios_en_los_barrios_de_nueva_creacion_de_la_ciudad_de_zaragoza}
