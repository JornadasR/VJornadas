\chapter{Estrategias de Captación de Clientes en Mercados con Competencia}

\chapterprecis{Francisco Jesús Rodríguez Aragón\\Doctor en Estadística por la Universidad de Códoba\\Associate Professional Risk Manager}

\index{Jesús Rodríguez Aragón, Francisco}

\index[inst]{Doctor en Estadística por la Universidad de Códoba}
\index[inst]{Associate Professional Risk Manager}

    En este trabajo se lleva a cabo un análisis del entorno competitivo de una empresa determinada junto con la elaboración de una estrategia de de búsqueda y optimización, geo-referenciada, de clientes teniendo en cuenta los siguientes hitos principales en su desarrollo:

-Localización de los competidores y el establecimiento de áreas geográficas de concentración
-Ubicación de nichos de mercado y definición de zonas de concentración de lo que se va a entender como mercado potencial
	    -Facilitar la toma de decisiones en cuanto a: 
		    -La realización o no de acciones comerciales
	 	    -Dónde realizar las anteriores acciones comerciales
    -La posibilidad de llevar a cabo campañas de publicidad y/o marketing (y de sus problemas derivados como localización de postes publicitarios, optimización del buzoneo, etc)

    El informe que aquí se presenta ofrece un Análisis de Prospección de Mercados con el que se ofrece un ejemplo de la potencialidad que se podría obtener del uso efectivo de bases de datos como SABI si se le suma la potencialidad del lenguaje R junto con análisis estadísticos en materia de riesgo y análisis de la competencia.
    Este trabajo está formado por un conjunto de 5 análisis interrelacionados cuya idea principal se basa en la interrelación de la competencia con el mercado potencial dado un determinado cliente, así pues, en el primer paso se procede a realizar un análisis general y relativo de tipo financiero del estatus de la industria y del sector competitivo considerado en sí, para posteriormente localizar de un modo segmentado a la competencia; tras estos pasos, en el tercero se define lo que se entiende por mercado potencial y cómo localizar nichos claves de nuevos clientes, de modo que en un siguiente paso lo se analiza es la distribución de dichos clientes, para finalmente en el último análisis, relacionar las concentraciones de clientes con las de empresas competitivas de modo más o menos segmentado en base a la calidad crediticia del mercado de un modo que finalmente se puedan tomar decisiones acertadas de actuación muy enfocadas al área marketing-comercial, pero manteniendo en todo momento el sentido clave del riesgo asociado a estos nuevos clientes que integran los mercados potenciales y que aquí se construyen y se analizan.
    Finalmente debe indicarse que el análisis que aquí se realiza va enfocado fundamentalmente a sociedades que publican (y en general tienen obligación de ello) información financiera excluyéndose a los autónomos y a aquellas sociedades que no la emiten

%\bibliographystyle{plain}

%\bibliography{resumenes/estrategias_de_captacion_de_clientes_en_mercados_con_competencia}
