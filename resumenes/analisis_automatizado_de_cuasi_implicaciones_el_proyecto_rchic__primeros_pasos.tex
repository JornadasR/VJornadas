\chapter{Análisis Automatizado de Cuasi-Implicaciones el Proyecto RCHIC: primeros pasos}

\chapterprecis{Rubén Pazmiño, Raphael Couturier, Pablo Gregori\\Escuela Superior Politécnica de Chimborazo. Ecuador\\Universida Comte. Francia\\Universidad Jaume I. España}

\index{Pazmiño, Rubén}
\index{Couturier, Raphael}
\index{Gregori, Pablo}

\index[inst]{Escuela Superior Politécnica de Chimborazo. Ecuador}
\index[inst]{Universida Comte. Francia}
\index[inst]{Universidad Jaume I. España}

El chic (por sus siglas en francés Classification HiérarchiqueImplicative et Cohésitive) es el único programa que permite hacer realidad los resultados teóricos del Análisis Estadístico Implicativo. Ésta teoría se ha desarrollado desde los años 70 por el profesor Régis Grasy colaboradores y permite determinar cuasi-implicaciones entre variables y clases de variables. En forma simplificada permite establecer reglas del tipo: Si se observa a, entonces se observa generalmente b. El software chic es un software propietario de origen francés, elaborado por Raphaël Couturier, que trabaja en la plataforma Windows, en 6 idiomas, con una interface sencilla, liviano y que permite los siguientes análisis: árboles de similaridad, grafo implicativo, árbol cohesitivo y reducción. Este trabajo tiene el objetivo de socializar el proyecto Rchic (chic libre basado en R) y sus avances. El proyecto Rchic consiste en diseñar un entorno colaborativo para elaborar una versión libre del software propietario chic basada en el lenguaje estadístico R. \bigskip\subsection*{Bibliografía}

 1. Regis, Grass. Contribution a l'etude experimentale et a l'analyse de certaines acquisitions cognitives et de certains objectifs didactiques en mathematiques. Rennes : These d'Etat, 1979.

2. Regnier, Jean Claude. 7? Coloquio Internacional de Analisis Estadistico Implicativo (ASI 7). [En linea] 18 de Junio de 2013. [Citado el: 9 de Septiembre de 2013.] http://sites.univ-lyon2.fr/asi7/.

3. Gras, Regis y Bailleul, Marc. Primer Coloquio Internacional de Analisis Estadistico Implicativo (ASI 1). [En linea] 23 de Junio de 2000. [Citado el: 9 de Septiembre de 2013.] http://math.unipa.it/~grim/asi/asi\_00\_CAEN.htm.

4. Spagnolo, Filippo. I COLOQUIO O METODO ESTATISTICO IMPLICATIVO UTILIZADO EM ESTUDOS QUALITATIVOS DE REGRAS DE ASSOCIACAO CONTRIBUICAO A PESQUISA EM EDUCACAO. [En linea] 9 de Julio de 2003. [Citado el: 9 de Septiembre de 2013.] http://math.unipa.it/~grim/asi/asi\_03\_brasil.htm.

5. --. Groupe International d'Analyse Statistique Implicative. Actes des Journees - Proceedings - Atti di Convegni. [En linea] 6 de Octubre de 2005. [Citado el: 10 de Septiembre de 2013.] http://math.unipa.it/~grim/asi/asi\_index.htm.

6. Regnier, Jean Claude. 6? Coloquio Internacional de Analisis Estadistico Implicativo (ASI6). [En linea] 7 de Noviembre de 2012. [Citado el: 13 de Septiembre de 2013.] http://sites.univ-lyon2.fr/asi6/.



%\bibliographystyle{plain}

%\bibliography{resumenes/analisis_automatizado_de_cuasi_implicaciones_el_proyecto_rchic__primeros_pasos}
