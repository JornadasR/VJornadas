\chapter{Preprocesado de imágenes hiperespectrales en R}

\chapterprecis{Rubén Urraca Valle, Borja Millán, Roberto Fernandez Martinez, Andres Sanz Garcia\\TELEVITIS. Universidad de La Rioja, Spain\\Department of Mining and Metallurgical Engineering and Materials Science. University of Basque Country, Spain\\Division of Bioscience. University of Helsinki, Finland}

\index{Urraca Valle, Rubén}
\index{Millán, Borja}
\index{Fernandez Martinez, Roberto}
\index{Sanz Garcia, Andres}

\index[inst]{TELEVITIS. Universidad de La Rioja, Spain}
\index[inst]{Department of Mining and Metallurgical Engineering and Materials Science. University of Basque Country, Spain}
\index[inst]{Division of Bioscience. University of Helsinki, Finland}

En la actualidad, el desarrollo de los sensores hiperespectrales está abriendo numerosas líneas de investigación. Estos sensores, a diferencia de las cámaras convencionales, son capaces de recoger información en múltiples frecuencias dando lugar a la generación de espectros [1]. Con los espectros el número de datos disponible se multiplica, dando lugar a la aparición de cubos de datos. Sin embargo, un análisis apropiado de los mismos permite identificar diversas propiedades de los materiales. Esto ha propiciado que las técnicas hiperespectrales se estén extendiendo a numerosos campo, desde la medicina a la agricultura pasando por la biología.
En esta comunicación se busca describir el proceso de importación y preprocesado de datos procedente de los sensores hiperespectrales a R dentro del sector agrícola. Para ello se trabajará con dos tipos de sensores: un sensor NIR puntual (microPHAZIR Analyzer), que genera un único espectro (vector de datos) y una cámara hiperescpectral que abarca tanto el rango NIR como el visible y genera un espectro por cada uno de los pixeles recogidos (cubo de datos). Los objetos tratados serán bayas de uva y hojas de diferentes variedades de cepa.
Tradicionalmente, los datos son extraídos de la cámara y preprocesados en software muy especializados proporcionados por el propio fabricante del sensor o en software comerciales como Matlab. Sin embargo, cuando se quiere pasar a la fase de postprocesado, se realiza una transferencia de datos a software más especializados en análisis y de mayor disponibilidad como R. En este trabajo se pretende importar directamente los datos desde el sensor a R, eliminando así el uso de software comercial. Para ello se analiza una de las librerías disponibles en R para el tratamiento de espectros, hyperSpec. El objetivo es importar los diferentes formatos generados por los sensores (.txt, .spc, .pdo …) y guardarlos como objetos hyperSpec para así facilitar la tarea de análisis. Una vez importados se procede al postprocesado de datos, siendo un proceso clave sobre todo en las imágenes de la cámara hiperespectral donde se dispone de más de 1 espectro. El proceso de postprocesado incluye los siguientes pasos: segmentación, eliminación de picos, eliminación de pixels muertos, aplicación de filtros, calibrado. Con este proceso se consiguen medidas robustas para la posterior fase de análisis sin la necesidad de utilizar software adicionales a R [2]. \bigskip\subsection*{Bibliografía}

 1. Grahn, H.F., Geladi, P. (2007) Techniques and applications of hyperspectral image analysis. Wiley

2. Wehrens, R. (2011) Chemmometrics with R. Springer

%\bibliographystyle{plain}

%\bibliography{resumenes/preprocesado_de_imagenes_hiperespectrales_en_r}
