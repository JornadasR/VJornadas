\chapter{Postprocesado de resultados de analysis de elementos finitos con R}

\chapterprecis{Andres Sanz-Garcia, Julio Fernandez-Ceniceros, Ruben Urraca-Valle, Roberto Fernandez-Martinez\\Division of Bioscience. University of Helsinki, Finland\\EDMANS. Universidad de La Rioja, Spain\\TELEVITIS. Universidad de La Rioja, Spain\\Department of Mining and Metallurgical Engineering and Materials Science. University of Basque Country, Spain}

\index{Sanz-Garcia, Andres}
\index{Fernandez-Ceniceros, Julio}
\index{Urraca-Valle, Ruben}
\index{Fernandez-Martinez, Roberto}

\index[inst]{Division of Bioscience. University of Helsinki, Finland}
\index[inst]{EDMANS. Universidad de La Rioja, Spain}
\index[inst]{TELEVITIS. Universidad de La Rioja, Spain}
\index[inst]{Department of Mining and Metallurgical Engineering and Materials Science. University of Basque Country, Spain}

Los avances en las técnicas de simulación numérica y el desarrollo de entornos GUI para el tratamiento de los datos de entrada/salida ha permitido la generación de modelos más realistas [1]. A pesar de ello, el proceso de simular requiere de una serie de detallados pasos que consumen mucho tiempo y recursos. R-project es un lenguaje de programación que ha crecido en flexibilidad y en usos. De hecho, la automatización de tareas para encaminadas a generar flujos de datos procesados es un campo con gran potencial.
Mediante el uso de distintos objetos y sus métodos englobados en librerías, R permite reducir los tiempos de procesamiento de repetidas simulaciones [2]. El proceso mediante la generación de scripts que engloban multiple tareas asociadas a cada paso. Algunas de ellas son la generación aleatoria los datos de entrada, ejecución de tareas o subrutinas, control de salidas y generación de gráficas, etc. En esta comunicación se describe un caso aplicado a la simulación de modelos de sólidos continuos mediante el uso del software ABAQUS[3] y el lenguaje de programación Python.

%\bibliographystyle{plain}

%\bibliography{resumenes/postprocesado_de_resultados_de_analysis_de_elementos_finitos_con_r}
