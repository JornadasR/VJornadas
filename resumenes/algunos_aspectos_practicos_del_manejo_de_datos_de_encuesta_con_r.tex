\chapter{Algunos aspectos prácticos del manejo de datos de encuesta con R}

\chapterprecis{Jesús Bouso Freijo\\Centro de Investigaciones Sociológicas (CIS)}

\index{Bouso Freijo, Jesús}

\index[inst]{Centro de Investigaciones Sociológicas (CIS)}

La presentación pretende ser un breve compendio de algunas herramientas útiles contenidas en diversos paquetes para el manejo de datos de encuesta. Fundamentalmente, las ideas a exponer proceden de la experiencia adquirida trabajando con R en el Centro de Investigaciones Sociológicas (CIS). Los datos de estudios del CIS cuentan con la particularidad de presentar una estructura variable que hace muy complicada la automatización sistemática del manejo de los mismos. También es relevante para su tratamiento con R la supremacía del programa SPSS en el ámbito de la Sociología, las Ciencias Políticas y otras disciplinas sociales afines. Por su parte, Stata va adquiriendo cierta presencia en estos ámbitos. Ello hace conveniente analizar las posibilidades que ofrece R a la hora de interactuar con datos de otros paquetes. Por otra parte, se presenta brevemente el modo en que la batería de series temporales publicada por el CIS denominada “Indicadores del Barómetro” se halla implementada en R. Por último, se introduce muy someramente el papel jugado hasta ahora por R en el tratamiento estándar de metadatos de encuestas.

En resumen, cabe citar como puntos principales a tratar los siguientes:

•Interacción con datos de otros paquetes estadísticos

•Interacción con bases de datos

•Ideas para la lectura de ficheros de estructura variable (como los estudios del CIS)

•Utilización de R en el CIS: Los Indicadores del Barómetro

•Metadatos con R: Data Documentation Initiative (DDI)

%\bibliographystyle{plain}

%\bibliography{resumenes/algunos_aspectos_practicos_del_manejo_de_datos_de_encuesta_con_r}
